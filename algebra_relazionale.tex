\section{Algebra relazionale}
Data una base di dati, dobbiamo essere in grado di \emph{interrogarla}. Interrogare una
base di dati significa accedere a determinate informazioni tramite uno specifico linguaggio
di interrogazione. Le specifiche essenziali di tale linguaggio sono formalizzate 
nell'\textbf{algebra relazionale}; essa consiste di un insieme di \emph{operatori} che 
possono essere applicati a una (operatore unario) o due (operatore binario) istanze di 
relazione e \emph{restituiscono} un'istanza di relazione. L'algebra relazionale è un 
\textbf{linguaggio procedurale}: l'interrogazione consiste in un'espressione in
cui compaiono \emph{operatori} e \emph{istanze di relazione} della base di dati.

\subsection{Proiezione}
La \emph{proiezione} è un operatore che consente di effettuare un ``taglio verticale''
su una relazione, cioè di selezionare dolo alcune colonne (attributi). Viene indicata
con il simbolo:
\begin{center}
\begin{math}
 \pi_{A_1, A_2, \ldots, A_k}(r)
\end{math}
\end{center}
e seleziona le colonne di $r$ che corrispondono agli attributi $A_1$, $A_2$, \ldots, $A_k$.

\begin{exmp}
Si consideri la seguente relazione $Cliente$:
\begin{center}
\begin{tabular}{l |l | l}
  Nome & C\# & Città\\
  \hline
  Rossi & C1 & Roma\\
  Rossi & C2 & Milano\\
  Bianchi & C3 & Roma\\
  Verdi & C4 & Roma\\
\end{tabular}
\end{center}
Interroghiamo la base di dati con l'operatore di proiezione nel seguente modo: $\pi_{Nome}(Cliente)$.
Verrà restituita la seguente istanza:
\begin{center}
\begin{tabular}{l}
  Nome \\
  \hline
  Rossi \\
  Bianchi\\
  Verdi\\
\end{tabular}
\end{center}
Da notare come siano stati eliminati i valori doppi poiché viene restituito un \emph{insieme}.
\end{exmp}

\subsection{Selezione}
L'operatore di \emph{selezione} consente di effettuare un ``taglio orizzontale'' su una relazione,
cioè di selezionare solo le righe (tuple) che soddisfano una data condizione. Si denota con il
simbolo
\begin{center}
 $\sigma_{C}(r)$
\end{center}
dove $C$ è detta condizione di selezione.\\
La \emph{condizione di selezione} è un'espressione booleana in cui i termini semplici 
sono del tipo
\begin{center}
 $A \Theta B$ \\ oppure \\ $A \Theta$`$a$' \\
\end{center}
dove:
\begin{itemize}
 \item $\Theta$ è un operatore di confronto $\Theta \in \{<, =, >, \leq, \geq\}$;
 \item $A$ e $B$ sono due attributi con lo stesso dominio, $dom(A) = dom(B)$;
 \item $a \in dom(A)$
\end{itemize}

\begin{exmp}
 Si riprenda in considerazione la seguente tabella:
 \begin{center}
\begin{tabular}{l |l | l}
  Nome & C\# & Città\\
  \hline
  Rossi & C1 & Roma\\
  Rossi & C2 & Milano\\
  Bianchi & C3 & Roma\\
  Verdi & C4 & Roma\\
\end{tabular}
\end{center}
Vogliamo ottenere i dati dei clienti che risiedono a Roma. La \emph{query}
si traduce in:\\$\sigma_{Citt\grave{a}=`Roma'}(Cliente)$. L'istanza di relazione che ci 
viene restituita è la seguente:
 \begin{center}
\begin{tabular}{l |l | l}
  Nome & C\# & Città\\
  \hline
  Rossi & C1 & Roma\\
  Bianchi & C3 & Roma\\
  Verdi & C4 & Roma\\
\end{tabular}
\end{center}
\end{exmp}
\begin{exmp}
Facendo riferimento alla tabella dell'esempio precendente, vogliamo ottenere i dati dei  
clienti che si chiamano ``Rossi'' e risiedono a Roma. La \emph{query}
si traduce in: \\$\sigma_{Citt\grave{a}=`Roma'\wedge Nome=`Rossi'}(Cliente)$. L'istanza 
di relazione che ci viene restituita è la seguente:
\begin{center}
\begin{tabular}{l |l | l}
  Nome & C\# & Città\\
  \hline
  Rossi & C1 & Roma\\
 \end{tabular}
 \end{center}
\end{exmp}

\subsection{Unione}
L'operatore di \emph{unione} consente di costruire una relazione contenente tutte
le tuple che appartengono ad almeno uno dei due operandi. Si denota con $r_1 \cup r_2$.
L'operazione di unione può essere applicata solo ad operandi \textbf{union compatibili},
cioè tali che:
\begin{itemize}
 \item hanno lo stesso numero di attributi;
 \item gli attributi corrispondenti sono definiti sullo stesso dominio.
\end{itemize}

\subsection{Differenza}
L'operatore di \emph{differenza} consente di costruire una relazione contenente tutte
le tuple del primo operando che non appartengono al secondo operando. L'operatore è
applicabile solo a operandi \emph{union compatibili}.

\subsection{Intersezione}
L'operatore di \emph{intersezione} consente di costruire una relazione contenente
tutte le tuple che appartengono ad entrambi gli operandi. Si denota con $r_1 \cap r_2 = 
(r_1 - (r_1 - r_2))$.

\subsection{Prodotto cartesiano}
L'operatore \emph{prodotto cartesiano} consente di costruire una relazione contenente
tutte le tuple che si ottengono concatenando una tupla del primo operando con una tupla
del secondo operando. Si denota con $r_1 \times r_2$.

\begin{exmp}
 Consideriamo le seguenti relazioni $Cliente$ e $Ordine$:
 \begin{center}
 \begin{multicols}{2}

 \begin{tabular}{l | l | l}
  Nome & C\# & Città\\
  \hline
  Rossi & C1 & Roma\\
  Rossi & C2 & Milano\\
 \end{tabular}
 
 \begin{tabular}{l | l | l}
  C\# & A\# & Num \\
  \hline
  C1 & A1 & 100\\
  C2 & A2 & 200\\
  C1 & A2 & 200\\
 \end{tabular}

\end{multicols}
\end{center}

Il prodotto cartesiano tra queste due relazioni da come risultato la relazione
che chiameremo per convenienza $Risultato$:

\begin{center}
 \begin{tabular}{l|l|l|l|l|l}
  Nome & C\# & Città & C\# & A\# & Num\\
  \hline
  Rossi & C1 & Roma & C1 & A2 & 100\\
  Rossi & C1 & Roma & C2 & A2 & 200\\
  Rossi & C1 & Roma & C1 & A2 & 200\\
  Rossi & C2 & Milano & C1 & A1 & 100\\
  Rossi & C2 & Milano & C2 & A2 & 200\\
  Rossi & C2 & Milano & C1 & A2 & 200\\
 \end{tabular}

\end{center}

\end{exmp}

\begin{exmp}
 Riprendendo la relazione $Risultato$ dall'esempio precendente, vogliamo effettuare
 una query che richieda i ``dati dei clienti e dei loro ordini'':
 \begin{center}
  \begin{math}
   \pi_{Nome,\ Cliente.C\#,\ Citt\grave{a},\ A\#,\ Num} 
   (\sigma_{Cliente.C\#\ =\ Ordine.C\#}(
   Cliente \times Ordine))
  \end{math}
 \end{center}
 La query viene eseguita in quest'ordine:
 \begin{enumerate}
  \item Viene effettuato il prodotto cartesiano $Cliente \times Ordine$;
  \item dal prodotto cartesiano vengono selezionate le tuple dove i corrispondenti
  attributi $C\#$ hanno lo stesso valore;
  \item sulla relazione ricavata dalla selezione (punto 2) viene operata una proiezione
  per prendere le colonne volute, che andranno a formare la relazione che
  sarà il risultato finale della query.
\begin{center}
 \begin{tabular}{l|l|l|l|l}
  Nome & C\# & Città  & A\# & Num\\
  \hline
  Rossi & C1 & Roma  & A2 & 100\\
  Rossi & C1 & Roma  & A2 & 200\\
  Rossi & C2 & Milano & A2 & 200\\
 \end{tabular}

\end{center}
 \end{enumerate}

\end{exmp}

\subsection{Join naturale}
L'operatore \emph{join naturale} consente di selezionare le tuple del prodotto cartesiano
dei due operandi che soddisfano la condizione
\begin{center}
\begin{math}
 (R_1.A_1 = R_2.A_1) \wedge (R_1.A_2 = R_2.A_2) \wedge \ldots \wedge (R_1.A_k = R_2.A_k)
\end{math}
\end{center}
Dove $R_1$ e $R_2$ sono i nomi delle due relazioni operando e $A_1 \ldots A_k$ sono gli 
attributi comuni.\\
Il join naturale è definito come
\begin{center}
\begin{math}
 r_1\bowtie r_2 = \pi_{XY}(\sigma_{C}(r_1 \times r_2))
\end{math}
\end{center}
dove
\begin{itemize}
 \item $C$ = $(R_1.A_1 = R_2.A_1) \wedge (R_1.A_2 = R_2.A_2) \wedge \ldots \wedge (R_1.A_k = R_2.A_k)$;
 \item $X$ sono gli attributi di $r_1$;
 \item $Y$ sono gli attributi di $r_2$ che non sono in $r_1$;
\end{itemize}

\subsection{Theta Join}
Il \emph{$\Theta-$Join} consente di selezionare le tuple del prodotto cartesiano
di due operandi che soddisfano una condizione del tipo $A$ $\Theta$ $B$, dove:
\begin{itemize}
 \item $\Theta$ è un operatore di confronto $\Theta \in \{<, =, >, \leq, \geq\}$;
 \item $A$ è un attributo dello schema del primo operando e $B$ uno del secondo;
 \item $dom(A) = dom(B)$;
\end{itemize}



