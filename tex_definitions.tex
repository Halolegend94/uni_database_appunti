%definiamo i colori
\definecolor{c_theo}{HTML}{FFF99C}
\definecolor{c_exmp}{HTML}{B9FF9D}
\definecolor{c_def}{HTML}{CEDAFF}
\definecolor{c_alg}{HTML}{DCB6FF}
\newtheoremstyle{mys}
  {\topsep}
  {\topsep}%
  {}
  {}%
  {\bfseries}{}%
  {\newline}{}%
\theoremstyle{mys}

%riquadro della proposizione
\newmdtheoremenv[%
roundcorner = 10pt,
linecolor=lightgray!30,
backgroundcolor=lightgray!30,%
innertopmargin=3pt]{prop}{Proposizione}[section]



%riquadro del teorema
\newmdtheoremenv[%
linecolor=c_theo!55,
backgroundcolor=c_theo!55,%
innertopmargin=3pt]{theo}{Teorema}[section]

%riquadro esempio
\newmdtheoremenv[%
linecolor=c_exmp!40,
backgroundcolor=c_exmp!40,%
innertopmargin=3pt]{exmp}{Esempio}[section]

%riquadro lemma
\newmdtheoremenv[%
linecolor=lightgray!30,
backgroundcolor=lightgray!30,%
innertopmargin=3pt,]{lem}{Lemma}[section]

%riquadro definizione
\newmdtheoremenv[%
linecolor=c_def!50,
backgroundcolor=c_def!50,%
innertopmargin=3pt]{defn}{Definizione}[section]

%riquadro codice
\newmdtheoremenv[%
linecolor= c_alg!30,
backgroundcolor=c_alg!30,%
innertopmargin=3pt]{alg}{Algoritmo}[section]