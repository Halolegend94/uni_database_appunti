\section{Esami svolti}
In questa sezione sono raccolti alcune tracce degli esami della professoressa
Moscardini con soluzioni. Tali soluzioni sono state elaborate con diversi studenti
e basate anche sulla correzione di alcuni esercizi della professoressa in aula e 
in sede di revivisione del compito. Prima di elencare tali soluzioni, vediamo quali
sono le tecniche per risolvere i principali e ricorrenti quesiti presenti in sede d'esame.\\
\textbf{NOTA}: le tracce dei compiti sono sempre le stesse da almeno un paio di anni, ma non
si può garantire che resteranno uguali negli anni a venire. Ciò non toglie che tali esercizi 
siano fondamentali per questo esame.
\subsection{Tecniche risolutive}
Il compito sarà diviso in tre parti:
\begin{enumerate}
 \item \textbf {Algebra relazionale}: vi verrà chiesto di esprimere in tale linguaggio
 un'interrogazione ad un database per ottenere particolari informazioni. Su questa prima parte
 non c'è molto da dire, si applicano semplicemente i vari operatori a seconda della richiesta; si useranno in 
 congiunzione la \emph{proiezione}, \emph{la selezione} e il molto 
 ``gettonato'' \textbf{join}. Vi verranno date tipicamente tre relazioni: il pattern che si ripete
 è che due relazioni non hanno nessun attributo in comune e rappresentano due ``realtà disgiunte'', mentre la terza
 ha due attributi (oltre a diversi altri) che rimandano alle chiavi delle altre due tabelle. In questo modo è possibile 
 mettere tutte e tre in relazione con il join (impostando tali attributi in comune come ``$\Theta$ del join'').
 \item \textbf {Terza forma normale}: il secondo esercizio chiede essenzialmente di applicare alcuni dei cinque
 algoritmi visti nella \textsc{Sezione 4} di questa dispensa per verificare delle proprietà. Tipica richiesta
 è dimostrare che un dato insieme di attributi è chiave per R, oppure vi sarà chiesto di trovare una scomposizione
 che preservi delle determinate proprietà. Le tecniche per risolvere questo tipo di esercizi sono ampiemente illustrate nei vari
 esempi nella sezione apposita, quindi non le ripeteremo qui. Unica nota è che quando vi si chiede se la relazione è in \emph{3NF}
 è consigliato utilizzare la \textsc{Definizione 4.6}.
 \item \textbf{Organizzazione fisica}: con la terza parte vuole si vuole testare se lo studente ha assimilato 
 in maniera adeguata i modi in cui può ossere organizzato fisicamente il database, quindi si consiglia di studiare bene la \textsc{Sezione 5}
\end{enumerate}

